


\documentclass[a4paper,12pt]{article}

\usepackage{cmap}					
\usepackage[T2A]{fontenc}			
\usepackage[utf8]{inputenc}			
\usepackage[english,german]{babel}	
\usepackage{caption}
\usepackage{indentfirst}


\author{...Koniev...}    %SCHREIBEN SIE BITTE HIER IHRE NACHNAMEN
\title{Aufgabenblatt 1}
\date{Abgabetermin: 14.05.2017 23:55 Uhr}

\begin{document} 

\maketitle

\section*{Aufgabe 1.2: Betriebssysteme (1,5 Punkte)}
 a) Welche drei Aufgabenbereiche und sieben Einzelfunktionen besitzt ein Betriebssystem? (0,5 Punkte)
 
\vspace{\baselineskip}
 \large Grobe Aufteilung in drei Aufgabenbereiche:
 
 \normalsize
 \begin{enumerate}
 	\item Bereitstellung von Hilfsmitteln für Benutzerprogramme
 	\item Vernachlässigung der genauen Benutzerkenntnis von HW-Eigenschaften und spezieller SW-Komponenten, wie z.B. Gerätetreiber
 	\item Koordination und Vergabe der zur Verfügung stehenden Betriebsmittel an mehrere, gleichzeitig arbeitende Benutzer
 \end{enumerate}


\vspace{\baselineskip}
\large Einzelfunktionen eines Betriebssystems:
\normalsize
\begin{enumerate}
	\item Unterbrechungsverarbeitung (interrupt handling)
	\item Verteilung (dispatching): Prozessumschaltung
	\item Betriebsmittelverwaltung (resource management): Belegen, Freigeben und Betreiben von Betriebsmitteln, Werkzeuge zur Prozesssynchronisation
	\item Programmallokation (program allocation): Linken von Teilprogrammen, Laden und Verdrängen von Programmen in/aus dem Hauptspeicher
	\item Dateiverwaltung (file mangement): 
	\begin{itemize}
		\item Organisation des SPeicherplatzes in Form von Dateien auf Datenträgern
		\item Bereitstellung von Funktionen zur Speicherung, Modifikation und Wiedergewinnung der gespeicherten Informationen
	\end{itemize}
	\item Auftragsteuerung (job congtrol): Festlegnung der REihenfolge, in der die eingegangegen Aufträge und dereen Bestandteile bearbeitet werden sollen
	\item Zuverläsigkeit (reliability):
	\begin{itemize}
		\item Funktionen zur Reaktion auf Störungen und Ausfälle der Rechnerhardware sowie auf Fehler in der Software
		\item Korrektheit, Robustheit und Toleranz (ständig betriebsbereit unter der Aufrechterhaltung einer Mindestfunktionsfähigkeit)
	\end{itemize}
\end{enumerate}

\vspace{\baselineskip}
\vspace{\baselineskip}
\vspace{\baselineskip}
\large Grundlegende Betriebssystemfunktionen:
\normalsize
\begin{enumerate}
	\item Dateiverwaltung (filemanagement)
	\begin{itemize}
		\item Organisation des Speicherplatzes in Form von Dateien auf Datenträgern
		\item Bereitstellung von Funktionen zur Speicherung, Modifikation und Wiedergewinnung der gespeicherten Informationen
	\end{itemize}
	\item Auftragsteuerung (jobcontrol)
	\begin{itemize}
		\item Festlegung der Reihenfolge, in der die eingegangenen Aufträge und deren Bestandteile bearbeitet werden sollen
	\end{itemize}
	\item Zuverlässigkeit (reliability)
	\begin{itemize}
		\item Funktionen zur Reaktion auf Störungen und Ausfälle der Rechnerhardware sowie auf Fehler in der Software
		\item Korrektheit, Robustheit und Toleranz (ständig betriebsbereit unter der Aufrechterhaltung einer Mindestfunktionsfähigkeit)
	\end{itemize}
\end{enumerate}

\vspace{\baselineskip}
 b) Was ist der Unterschied zwischen Mechanismen und Policies? Nennen Sie je zwei Beispiele für diese.
(0,5 Punkte)

\vspace{\baselineskip}

 \large Wichtige Unterscheidung zwischen Mechanismen und Policies:
 \normalsize
 	Ein Mechanismus ist quasi analog zu einer Interface in Java, in dem es sagt was für Funktionalität angefordert wird; Policies sind eher wie dies implimentiert wird, analog zur Implimentation einer Interface in Java.
	\begin{enumerate}
		\item Mechanismus: Wie wird eine Aufgabe prinzipiell gelöst?
		\item Policy: Welche Vorgaben/Parameter werden im konkreten Fall eingesetzt?


	\end{enumerate}

\vspace{\baselineskip}

\large Beispiele:
\normalsize
\begin{enumerate}
	\item Zeitscheibenprinzip
	\begin {itemize}
		\item Existenz eines Timerszur Bereitstellung von Unterbrechungen $\Rightarrow$ Mechanismus
		\item Entscheidung, wie lange die entsprechende Zeit für einzelne Anwendungen / Anwendungsgruppen eingestellt wird $\Rightarrow$ Policy

		\item Es werden Daten, die in der Nähe gespeichert sind, mit in den Cache gebracht bzw. räumliche Lokalität wird ausgenutzt $\Rightarrow$ Mechanismus
		\item Entscheidung, wie viele auf jeder Seite des ausgewählten Dateis genommen werden und was sie ersetzten sollten $\Rightarrow$ Policy
	\end{itemize}
\vspace{\baselineskip}
\vspace{\baselineskip}
	\item Trennung wichtig für Flexibilität
	\begin {itemize}
		\item Policies ändern sich im Laufe der Zeit oder bei unterschiedlichen Plattformen $\Rightarrow$  Falls keine Trennung vorhanden, muss jedes Mal auch der grundlegende Mechanismus geändert werden
		\item Wünschenswert: Genereller Mechanismus, so dass eine Policiesveränderung durch Anpassung von Parametern umgesetzt werden kann
	\end{itemize}
\end{enumerate}

\vspace{\baselineskip}
c) Nennen Sie die fünf häufigsten Designstrukturen für Betriebssysteme, und beschreiben Sie eine davon
im Detail. (0,5 Punkte)
\vspace{\baselineskip}

\large Häufige Designstrukturen für Betriebssysteme:
\normalsize
\begin{enumerate}
	\item Monolithische Systeme
	\item Geschichtete Systeme
	\item Virtuelle Maschinen
	\item Exokern
	\item Client-Server-Systeme mit Mikrokern
\end{enumerate}

 \vspace{\baselineskip}
 \large Monolithische Systeme: Häufigste Realisierungsform
 \normalsize
 \begin{enumerate}
 	\item Große Menge von Funktionen mit wohl-definierten Schnittstellen für Parameter und Ergebnisse $\Rightarrow$ Alle Funktionen bilden den Objektcode
 	\item Hauptprogramm: Ruft die Dienstfunktionen auf
 	\item Dienstfunktionen: Führen die Systemaufrufe durch
 	\item Hilfsfunktionen: stellen Mechanismen bereit, die von diesen benötigt werden, z.B. Kopieren von Daten aus einem Programm
 \end{enumerate}


\section*{Aufgabe 1.1: Logischer Adressraum (0,5 Punkte)}

	In der Vorlesung haben Sie das Layout des logischen Adressraums kennengelernt. Nachfolgend finden Sie
den Code für ein einfaches Programm. Skizzieren Sie für die vier markierten Stellen jeweils den Aufbau
des Stacks.
\begin{table}
	\begin{center}
		\begin{tabular}{|c|}
	\\ 
	\hline main() \\ int x = 5;
	\\ 
	\hline 
		\end{tabular}
	\end{center}
 \caption*{<<Stelle 1>>}
\end{table}
\vspace{\baselineskip}
\vspace{\baselineskip}


\begin{table}
	\begin{center}
		\begin{tabular}{|c|}
			\\ 
			\hline sum(0) \\ n = 0 \\ return 0;\\
			\hline sum(1) \\ n = 1 \\ return sum(0) + 1;\\
			\hline sum(2) \\ n = 2 \\ return sum(1) + 2;\\
			\hline sum(3) \\ n = 3 \\ return sum(2) + 3;\\
			\hline sum(4) \\ n = 4 \\ return sum(3) + 4;\\
			\hline sum(5) \\ n = 5 \\ return sum(4) + 5;\\
			\hline main() \\ int x = 5;\\
			\hline 
		\end{tabular}
	\end{center}
	\caption*{<<Stelle 2>>}
\end{table}

\vspace{\baselineskip}
\vspace{\baselineskip}
\vspace{\baselineskip}
\begin{table}
	\begin{center}
		\begin{tabular}{|c|}
			\\ 
			\hline sum\_direkt(5) \\ return 15;\\
			\hline main() \\ int x = 5;	\\ 
			\hline 
		\end{tabular}
	\end{center}
	\caption*{<<Stelle 3>>}
\end{table}

\vspace{\baselineskip}

\begin{table}
	\begin{center}
		\begin{tabular}{|c|}
			\\ 
			\hline main() \\ int x = 5; \\ ergebnis = 15; \\ ergebnis = 15; \\ return 0;\\
			\hline 
		\end{tabular}
	\end{center}
	\caption*{<<Stelle 4>>}
\end{table}

\clearpage
\section*{Quelle:}

O. Kao: Systemprogrammierung 2017 <<Folie 1>>
  
 
 
 




\end{document} 

